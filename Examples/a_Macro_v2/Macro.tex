
%----jess----COMMON_PACKAGES----------------------------------------------------VVVV


\usepackage{geometry}       		% See geometry.pdf to learn the layout options.
\geometry{a4paper}                  	% ... or a4paper or a5paper or ... 

\usepackage[english]{babel} 		% Paquete de idioma
\usepackage{boiboites}			% Paquete para los recuadros de "teorema"
\usepackage{booktabs} 			% To thicken table lines
\usepackage{caption}			% To use captions without figure or table environment
\usepackage{multicol}			% For using multicols environment
\usepackage{verbatim}			% For verbatim environment
\usepackage{place ins}			% To use \FloatBarrier
\usepackage{color}				% To use colors
\usepackage{cancel}				% To use \calcel and others (cancel equations)

%\geometry{landscape}                	% Activate for rotated page geometry
%\usepackage[parfill]{parskip}    	% Activate to begin paragraphs with an empty line rather 								% than an indent
\usepackage{graphicx}
\usepackage{amssymb}
\usepackage{epstopdf}
\DeclareGraphicsRule{.tif}{png}{.png}{`convert #1 `dirname #1`/`basename #1 .tif`.png}

\usepackage{tcolorbox}			% For fancy boxes similar to theorem boxes
\tcbuselibrary{breakable,skins}		% Required by some features of tcolorbox
\usetikzlibrary{shadings}			% Required by some features of tcolorbox

\usepackage{lipsum}				% Fills with "random text" for tryouts
\usepackage{verbatim}			% Enables the comment environment
\usepackage{fancyhdr}			% For fancy header and footer
\usepackage{changepage}		% For reduced width

%----jess----COMMON_PACKAGES------------------------------------------------------^^^^



%----jess----PAGE_MARGINS--------------------------------------------------------------VVVV
\setlength{\textheight }{670pt}		% Modifies text height
\setlength{\voffset }{-15pt}		% Modifies a height parameter
\setlength{\textwidth }{460pt}		% Modifies text width
\setlength{\oddsidemargin }{-5pt}	% Modifies text width
%----jess----PAGE_MARGINS---------------------------------------------------------------^^^^


%----jess----THEOREM_BOX-----------------------------------------------------------------VVVV

\newboxedtheorem[boxcolor=black,background=white,titlebackground=blue!20,titleboxcolor=black]{Ejercicio}{}{}

%----jess----THEOREM_BOX-----------------------------------------------------------------^^^^


%----jess----FANCY_BOXES-----------------------------------------------------------------VVVV
% Box with colours and Title. Breakable across several pages
\newtcolorbox{Fancybox}[2][]{%
     	enhanced,
	colback=blue!0!white,
	%colframe=orange,
	top=4mm,
	bottom=4mm,
     	enlarge top by=\baselineskip/2+1mm,
     	enlarge top at break by=0mm,
	pad at break=2mm,
     	fontupper=\normalsize,			% Could use other size
	%label={#3},					% as in float figures. Req. updating no. of param. 2->3
	breakable,					% To be breakable across several pages.
	overlay unbroken and first={%
		\node[	rectangle,
				rounded corners,
				draw=black,
				fill=blue!10!white,
         			inner sep=1mm,
				anchor=west,
				%font=\small
			]
        	at ([xshift=4.5mm]frame.north west)
		{\strut\textbf{
				%Example 		% Name of the "boxes" series
				%\thetcbcounter: 	% If a counter is used
				#2				% Title
				}
		};
	},
	#1
}%
%----jess----FANCY_BOXES-----------------------------------------------------------------^^^^


%----jess----COMMANDS--------------------------------------------------------------------VVVV
\newcommand{\superscript}[1]{\ensuremath{^{\textrm{#1}}}}
\newcommand{\subscript}[1]{\ensuremath{_{\textrm{#1}}}}

% Arrays
\newcommand{\jarp}[2]{\left( \begin{array}{#1} #2 \end{array}\right)}
\newcommand{\jarc}[2]{\left[ \begin{array}{#1} #2 \end{array}\right]}

% Separator
\newcommand{\jacseparator}{		\vspace{0.4cm}		
\begin{center}	
\rule{1cm}{0.1pt} \\		\rule{2 cm}{0.2pt} \\		\rule{1cm}{0.1pt}
\end{center}					\vspace{0.4cm}		}



% insert a centered figure with caption and description AND WIDTH
% parameters 1:filename, 2:title, 3:description and label, 4: textwidth
% textwidth 1 means as text, 0.5 means half the width of the text
\newcommand{\figuremacroJ}[5]{
	\begin{figure}[htbp]
		\centering
		\includegraphics[width=#4\textwidth]{#1}
		\caption[#2]{\textbf{#2} #3}
		\label{#5}
	\end{figure}
}

\newcommand{\figmacroJ}[5]{
	\begin{figure}[htbp]
		\centering
		\includegraphics[scale=#4]{#1}
		\caption[#2]{\textbf{#2} #3}
		\label{#5}
	\end{figure}
}


\newcommand{\figmacroubicJ}[6]{
	\begin{figure}[#6]
		\centering
		\includegraphics[scale=#4]{#1}
		\caption[#2]{\textbf{#2} #3}
		\label{#5}
	\end{figure}
}


%Macro to include an image in a specific position (Not Float)
\newcommand{\graphicsmacroJ}[5]{
	\noindent
	\ \\
	\begin{minipage}{\textwidth}
		\centering
		\includegraphics[scale=#4]{#1}
		\captionof{figure}[#2]{\textbf{#2} #3} \label{#5}
	\end{minipage}
	
	\ \\
}

%Macro to include an image en a specific position with text next to it
%The text is introduced in #1 position 
\newcommand{\TgraphicsmacroJ}[6]{

	\noindent
	\begin{minipage}{\textwidth}
		#1  \\ \par
		\centering
		\includegraphics[scale=#5]{#2}
	\captionof{figure}[#3]{\textbf{#3} #4} \label{#6}
	\end{minipage}
	\ \\
}


%Macro for partial derivatives
\newcommand{\jpar}[2]{
	\frac{\partial{#1}}{\partial{#2}}
}

%Macro for equations
\newcommand{\jbe}[1]{
	\begin{equation}
	#1
	\end{equation}
}

%----jess----COMMANDS--------------------------------------------------------------------^^^^



%----unko----FIGURES-----------------------------------------------------------------------VVVV
% inserts a figure with wrapped around text; only suitable for NARROW figs
% o is for outside on a double paged document; others: l, r, i(inside)
% text and figure will each be half of the document width
% note: long captions often crash with adjacent content; take care
% in general: above 2 macro produce more reliable layout
\newcommand{\figuremacroN}[3]{
	\begin{wrapfigure}{o}{0.5\textwidth}
		\centering
		\includegraphics[width=0.48\textwidth]{#1}
		\caption[#2]{{\small\textbf{#2} - #3}}
		\label{#1}
	\end{wrapfigure}
}

%----unko----FIGURES-----------------------------------------------------------------------^^^^

%----mutt----FIGURES-----------------------------------------------------------------------VVVV

%El siguiente macro sirve para darle un cuadro tipo teorema a algun texto
%utilizado en ejercicios resueltos.
%Solo debe ponerse el texto dentro del parentesis despues de \cuadtext{}
\newcommand{\cuadtext}[2]{
    \begin{figure}[htbp]
    \begin{theo #1}
    #2
    \end{theo #1}
    \end{figure}
}

%El siguiente macro es utilizado en conjunto con el anterior
%sirve para agregar imagenes dentro de cuadros de texto
%notar que es el mismo que \figmacroJ solo que sin el ambiente
%figure que tiene conflicto al usar el ambiente theo
\newcommand{\figmacroCT}[5]{	
		\centering
		\includegraphics[scale=#4]{#1}
		\caption[#2]{\textbf{#2} #3}
		\label{#5}
}

%----mutt----FIGURES-----------------------------------------------------------------------^^^^

%---- jess----ADITIONAL_FILE_SPECIFIC---------------------------------------------VVVV

% Declaration of a question type

\newtcolorbox{Question}[2][]{
	colback=blue!0!white,		% 	colback=blue!5!white,
	colframe=blue!75!black,
	center title, 				enhanced,			
	shadow={2mm}{-1mm}{0mm}{black!50!white},
	title=\Large #2
}

% Declaration of a question type


% Declaration of new Shades
\newtcbox{\xmybox}[1][blue]{	nobeforeafter,			tcbox raise base, 
						arc=7pt,				colback=#1!10!white,
						colframe=#1!50!black,	
						before upper={\rule[-3pt]{0pt}{10pt}},
						boxrule=1pt,			boxsep=0pt,		
						left=6pt,				right=6pt,
						top=2pt,				bottom=2pt}
% Declaration of new Shades

% New Fancybox

% Box with colours and Title. Breakable across several pages
\newtcolorbox{Fancybox2}[2][]{%
     	enhanced,
	colback=blue!0!white,
	%colframe=orange,
	top=4mm,
	bottom=4mm,
     	enlarge top by=\baselineskip/2+1mm,
     	enlarge top at break by=0mm,
	pad at break=2mm,
     	fontupper=\normalsize,			% Could use other size
	%label={#3},					% as in float figures. Req. updating no. of param. 2->3
	breakable,					% To be breakable across several pages.
	overlay unbroken and first={%
		\node[	rectangle,
				rounded corners=7pt,
				%arc=7pt,
				draw=black,
				fill=blue!10!white,
         			inner sep=1mm,
				%anchor=west,
				anchor=center,
				%font=\small
			]
        	%at ([xshift=4.5mm]frame.north west)
		at (frame.north)
		{\strut\textbf{
				%Example 		% Name of the "boxes" series
				%\thetcbcounter: 	% If a counter is used
				#2				% Title
				}
		};
	},
	#1
}%

% New Fancybox

% White box
\newtcolorbox{whitebox_notcentred}[1]{
	enhanced, 			colback=blue!0!white,
	left=0mm,right=0mm, 	width=\textwidth,
	top=0mm,
	bottom=0mm,
	%colframe=blue!0!white,
	#1}

\newenvironment{whitebox}  
{ \begin{center}	\begin{whitebox_notcentred}{}} 
{ \end{whitebox_notcentred}  \end{center}}


%personalized width

\newenvironment{mywidth}{\begin{adjustwidth}{2cm}{2cm}}{\end{adjustwidth}}


%---- jess----ADITIONAL_FILE_SPECIFIC---------------------------------------------^^^^

\newcommand{\langl}{\begin{picture}(4.5,7)
\put(1.1,2.5){\rotatebox{60}{\line(1,0){5.5}}}
\put(1.1,2.5){\rotatebox{300}{\line(1,0){5.5}}}
\end{picture}}
\newcommand{\rangl}{\begin{picture}(4.5,7)
\put(.9,2.5){\rotatebox{120}{\line(1,0){5.5}}}
\put(.9,2.5){\rotatebox{240}{\line(1,0){5.5}}}
\end{picture}}






%%% Local Variables: 
%%% mode: latex
%%% TeX-master: "~/Documents/LaTeX/CUEDThesisPSnPDF/thesis"
%%% End: 
