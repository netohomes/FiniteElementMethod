\documentclass[letterpaper,10pt]{article}

\usepackage{bm}
\usepackage{amsfonts}
\usepackage{amsmath}


%----jess----COMMON_PACKAGES----------------------------------------------------VVVV


\usepackage{geometry}       		% See geometry.pdf to learn the layout options.
\geometry{a4paper}                  	% ... or a4paper or a5paper or ... 

\usepackage[english]{babel} 		% Paquete de idioma
\usepackage{boiboites}			% Paquete para los recuadros de "teorema"
\usepackage{booktabs} 			% To thicken table lines
\usepackage{caption}			% To use captions without figure or table environment
\usepackage{multicol}			% For using multicols environment
\usepackage{verbatim}			% For verbatim environment
\usepackage{place ins}			% To use \FloatBarrier
\usepackage{color}				% To use colors
\usepackage{cancel}				% To use \calcel and others (cancel equations)

%\geometry{landscape}                	% Activate for rotated page geometry
%\usepackage[parfill]{parskip}    	% Activate to begin paragraphs with an empty line rather 								% than an indent
\usepackage{graphicx}
\usepackage{amssymb}
\usepackage{epstopdf}
\DeclareGraphicsRule{.tif}{png}{.png}{`convert #1 `dirname #1`/`basename #1 .tif`.png}

\usepackage{tcolorbox}			% For fancy boxes similar to theorem boxes
\tcbuselibrary{breakable,skins}		% Required by some features of tcolorbox
\usetikzlibrary{shadings}			% Required by some features of tcolorbox

\usepackage{lipsum}				% Fills with "random text" for tryouts
\usepackage{verbatim}			% Enables the comment environment
\usepackage{fancyhdr}			% For fancy header and footer
\usepackage{changepage}		% For reduced width

%----jess----COMMON_PACKAGES------------------------------------------------------^^^^



%----jess----PAGE_MARGINS--------------------------------------------------------------VVVV
\setlength{\textheight }{670pt}		% Modifies text height
\setlength{\voffset }{-15pt}		% Modifies a height parameter
\setlength{\textwidth }{460pt}		% Modifies text width
\setlength{\oddsidemargin }{-5pt}	% Modifies text width
%----jess----PAGE_MARGINS---------------------------------------------------------------^^^^


%----jess----THEOREM_BOX-----------------------------------------------------------------VVVV

\newboxedtheorem[boxcolor=black,background=white,titlebackground=blue!20,titleboxcolor=black]{Ejercicio}{}{}

%----jess----THEOREM_BOX-----------------------------------------------------------------^^^^


%----jess----FANCY_BOXES-----------------------------------------------------------------VVVV
% Box with colours and Title. Breakable across several pages
\newtcolorbox{Fancybox}[2][]{%
     	enhanced,
	colback=blue!0!white,
	%colframe=orange,
	top=4mm,
	bottom=4mm,
     	enlarge top by=\baselineskip/2+1mm,
     	enlarge top at break by=0mm,
	pad at break=2mm,
     	fontupper=\normalsize,			% Could use other size
	%label={#3},					% as in float figures. Req. updating no. of param. 2->3
	breakable,					% To be breakable across several pages.
	overlay unbroken and first={%
		\node[	rectangle,
				rounded corners,
				draw=black,
				fill=blue!10!white,
         			inner sep=1mm,
				anchor=west,
				%font=\small
			]
        	at ([xshift=4.5mm]frame.north west)
		{\strut\textbf{
				%Example 		% Name of the "boxes" series
				%\thetcbcounter: 	% If a counter is used
				#2				% Title
				}
		};
	},
	#1
}%
%----jess----FANCY_BOXES-----------------------------------------------------------------^^^^


%----jess----COMMANDS--------------------------------------------------------------------VVVV
\newcommand{\superscript}[1]{\ensuremath{^{\textrm{#1}}}}
\newcommand{\subscript}[1]{\ensuremath{_{\textrm{#1}}}}

% Arrays
\newcommand{\jarp}[2]{\left( \begin{array}{#1} #2 \end{array}\right)}
\newcommand{\jarc}[2]{\left[ \begin{array}{#1} #2 \end{array}\right]}

% Separator
\newcommand{\jacseparator}{		\vspace{0.4cm}		
\begin{center}	
\rule{1cm}{0.1pt} \\		\rule{2 cm}{0.2pt} \\		\rule{1cm}{0.1pt}
\end{center}					\vspace{0.4cm}		}



% insert a centered figure with caption and description AND WIDTH
% parameters 1:filename, 2:title, 3:description and label, 4: textwidth
% textwidth 1 means as text, 0.5 means half the width of the text
\newcommand{\figuremacroJ}[5]{
	\begin{figure}[htbp]
		\centering
		\includegraphics[width=#4\textwidth]{#1}
		\caption[#2]{\textbf{#2} #3}
		\label{#5}
	\end{figure}
}

\newcommand{\figmacroJ}[5]{
	\begin{figure}[htbp]
		\centering
		\includegraphics[scale=#4]{#1}
		\caption[#2]{\textbf{#2} #3}
		\label{#5}
	\end{figure}
}


\newcommand{\figmacroubicJ}[6]{
	\begin{figure}[#6]
		\centering
		\includegraphics[scale=#4]{#1}
		\caption[#2]{\textbf{#2} #3}
		\label{#5}
	\end{figure}
}


%Macro to include an image in a specific position (Not Float)
\newcommand{\graphicsmacroJ}[5]{
	\noindent
	\ \\
	\begin{minipage}{\textwidth}
		\centering
		\includegraphics[scale=#4]{#1}
		\captionof{figure}[#2]{\textbf{#2} #3} \label{#5}
	\end{minipage}
	
	\ \\
}

%Macro to include an image en a specific position with text next to it
%The text is introduced in #1 position 
\newcommand{\TgraphicsmacroJ}[6]{

	\noindent
	\begin{minipage}{\textwidth}
		#1  \\ \par
		\centering
		\includegraphics[scale=#5]{#2}
	\captionof{figure}[#3]{\textbf{#3} #4} \label{#6}
	\end{minipage}
	\ \\
}


%Macro for partial derivatives
\newcommand{\jpar}[2]{
	\frac{\partial{#1}}{\partial{#2}}
}

%Macro for equations
\newcommand{\jbe}[1]{
	\begin{equation}
	#1
	\end{equation}
}

%----jess----COMMANDS--------------------------------------------------------------------^^^^



%----unko----FIGURES-----------------------------------------------------------------------VVVV
% inserts a figure with wrapped around text; only suitable for NARROW figs
% o is for outside on a double paged document; others: l, r, i(inside)
% text and figure will each be half of the document width
% note: long captions often crash with adjacent content; take care
% in general: above 2 macro produce more reliable layout
\newcommand{\figuremacroN}[3]{
	\begin{wrapfigure}{o}{0.5\textwidth}
		\centering
		\includegraphics[width=0.48\textwidth]{#1}
		\caption[#2]{{\small\textbf{#2} - #3}}
		\label{#1}
	\end{wrapfigure}
}

%----unko----FIGURES-----------------------------------------------------------------------^^^^

%----mutt----FIGURES-----------------------------------------------------------------------VVVV

%El siguiente macro sirve para darle un cuadro tipo teorema a algun texto
%utilizado en ejercicios resueltos.
%Solo debe ponerse el texto dentro del parentesis despues de \cuadtext{}
\newcommand{\cuadtext}[2]{
    \begin{figure}[htbp]
    \begin{theo #1}
    #2
    \end{theo #1}
    \end{figure}
}

%El siguiente macro es utilizado en conjunto con el anterior
%sirve para agregar imagenes dentro de cuadros de texto
%notar que es el mismo que \figmacroJ solo que sin el ambiente
%figure que tiene conflicto al usar el ambiente theo
\newcommand{\figmacroCT}[5]{	
		\centering
		\includegraphics[scale=#4]{#1}
		\caption[#2]{\textbf{#2} #3}
		\label{#5}
}

%----mutt----FIGURES-----------------------------------------------------------------------^^^^

%---- jess----ADITIONAL_FILE_SPECIFIC---------------------------------------------VVVV

% Declaration of a question type

\newtcolorbox{Question}[2][]{
	colback=blue!0!white,		% 	colback=blue!5!white,
	colframe=blue!75!black,
	center title, 				enhanced,			
	shadow={2mm}{-1mm}{0mm}{black!50!white},
	title=\Large #2
}

% Declaration of a question type


% Declaration of new Shades
\newtcbox{\xmybox}[1][blue]{	nobeforeafter,			tcbox raise base, 
						arc=7pt,				colback=#1!10!white,
						colframe=#1!50!black,	
						before upper={\rule[-3pt]{0pt}{10pt}},
						boxrule=1pt,			boxsep=0pt,		
						left=6pt,				right=6pt,
						top=2pt,				bottom=2pt}
% Declaration of new Shades

% New Fancybox

% Box with colours and Title. Breakable across several pages
\newtcolorbox{Fancybox2}[2][]{%
     	enhanced,
	colback=blue!0!white,
	%colframe=orange,
	top=4mm,
	bottom=4mm,
     	enlarge top by=\baselineskip/2+1mm,
     	enlarge top at break by=0mm,
	pad at break=2mm,
     	fontupper=\normalsize,			% Could use other size
	%label={#3},					% as in float figures. Req. updating no. of param. 2->3
	breakable,					% To be breakable across several pages.
	overlay unbroken and first={%
		\node[	rectangle,
				rounded corners=7pt,
				%arc=7pt,
				draw=black,
				fill=blue!10!white,
         			inner sep=1mm,
				%anchor=west,
				anchor=center,
				%font=\small
			]
        	%at ([xshift=4.5mm]frame.north west)
		at (frame.north)
		{\strut\textbf{
				%Example 		% Name of the "boxes" series
				%\thetcbcounter: 	% If a counter is used
				#2				% Title
				}
		};
	},
	#1
}%

% New Fancybox

% White box
\newtcolorbox{whitebox_notcentred}[1]{
	enhanced, 			colback=blue!0!white,
	left=0mm,right=0mm, 	width=\textwidth,
	top=0mm,
	bottom=0mm,
	%colframe=blue!0!white,
	#1}

\newenvironment{whitebox}  
{ \begin{center}	\begin{whitebox_notcentred}{}} 
{ \end{whitebox_notcentred}  \end{center}}


%personalized width

\newenvironment{mywidth}{\begin{adjustwidth}{2cm}{2cm}}{\end{adjustwidth}}


%---- jess----ADITIONAL_FILE_SPECIFIC---------------------------------------------^^^^

\newcommand{\langl}{\begin{picture}(4.5,7)
\put(1.1,2.5){\rotatebox{60}{\line(1,0){5.5}}}
\put(1.1,2.5){\rotatebox{300}{\line(1,0){5.5}}}
\end{picture}}
\newcommand{\rangl}{\begin{picture}(4.5,7)
\put(.9,2.5){\rotatebox{120}{\line(1,0){5.5}}}
\put(.9,2.5){\rotatebox{240}{\line(1,0){5.5}}}
\end{picture}}






%%% Local Variables: 
%%% mode: latex
%%% TeX-master: "~/Documents/LaTeX/CUEDThesisPSnPDF/thesis"
%%% End: 




% ----------------------------------------------------------------------------------------------------------------
%Jc : Jacob comment
%Proposed, diff command for derivatives:
\newcommand{\diff}[2]{ 
\frac{\mathrm{d}#1}{\mathrm{d}#2} 
}
%Example:
% $TEST:  \diff{^2 f(x)}{x} $



% ----------------------------------------------------------------------------------------------------------------




\begin{document}

%========================================================================================
%	1D EXAMPLES
%========================================================================================
\section{1D examples}

\section{Second order differential equation}

\subsection{Producing our differential equation}

Let's produce our own differential equation so we know the exact solution. Let's try with the following polynomial

\begin{equation}
f(x) = (x-1)(x-2)(x-3) = x^3 - 6x^2 + 11x - 6
\label{eq:SecondOrderEq_SolExact}
\end{equation}

Let's get the firs two derivates

\[
\begin{array}{rl}
f'(x) = & 3x^2 - 12x + 11 \\
f''(x) = & 6x - 12x
\end{array}
\]

And our second order differential equation is

\begin{equation}
\frac{d^2 f(x)}{dx^2} = 6x - 12x
\label{eq:SecondOrderEq}
\end{equation}

This form is known as the \textbf{strong} form of the partial differential equation. Of course, later we'll have to proportionate the appropriate boundary conditions to get the function shown in \ref{eq:SecondOrderEq_SolExact}. Then let's say our problem is





\begin{equation}
\begin{array}{rl}
\displaystyle
\frac{d^2 f(x)}{dx^2} = 6x - 12x & \textrm{ in } \mathbb{R} \\ \\
\displaystyle
\diff{^2 f(x)}{x^2} = 6x - 12x & \textrm{ in } \mathbb{R} \\ \\
\displaystyle
f(x) = 0 & \textrm{ on } x = 1 \\ \\
\displaystyle
f'(x) n_x = \frac{d f(x)}{d x} n_x = 2 & \textrm{ on } x = 3 
\end{array}
\label{eq:Problem_StrongForm}
\end{equation}


\graphicsmacroJ{figures/1_01_Objective_function.pdf}{Example's objective solution}{}{0.4}{fig:solution_equation}



\subsection{The weak form}

In order to use the finite element method to solve this equation, the first step is to write the differential equation in the \textbf{weak} or \textbf{variational} form. To do so, we require the Green's Theorem in 1D

\[
\int_{\Omega} \frac{\partial}{\partial x} \frac{\partial u}{\partial x} \, w \, d\Omega + \int_{\Omega} \frac{\partial u}{\partial x} \frac{\partial w}{\partial x} \, d\Omega = \int_{\Gamma} \frac{\partial u}{\partial x} n_x \, w \, d\Gamma
\]

Because we'll be working in 1D, then we can use $d$ instead of $\partial$
%Jc: What does each of them mean? 



\begin{equation}
\int_{\Omega} \frac{d}{d x} \frac{d u}{d x} \, w \, d\Omega + \int_{\Omega} \frac{d u}{d x} \frac{d w}{d x} \, d\Omega = \int_{\Gamma_D} \frac{d u}{d x} n_x \, w \, d\Gamma + \int_{\Gamma_N} \frac{d u}{d x} n_x \, w \, d\Gamma
\label{eq:GreenTheorem_1D}
\end{equation}

As you can see the boundary $\Gamma$ was split into $\Gamma_D$ and $\Gamma_N$, which are the places where the \textbf{essential} (in this case Dirichlet) % Jc: maybe, "also called Dirichlet"
and \textbf{natural} (in this case Neumann) %Jc. Same here, maybe
 boundary conditions, respectively, are imposed. $w$ is known as \textbf{test function} and works as a weight to do an average version of the original (the strong form) differential equation.

Using the equations \ref{eq:GreenTheorem_1D} and \ref{eq:SecondOrderEq} and in our case $u = f(x)$

\[
\int_{\Omega} (6x-12) w \, d\Omega + \int_{\Omega} \frac{d f(x)}{d x} \frac{d w}{d x} \, d\Omega = \int_{\Gamma_D} \frac{d f(x)}{d x} n_x \, w \, d\Gamma + \int_{\Gamma_N} \frac{d f(x)}{d x} n_x \, w \, d\Gamma
\]

Reordering

\[
\int_{\Omega} \frac{d f(x)}{d x} \frac{d w}{d x} \, d\Omega + \int_{\Omega} (6x-12) w \, d\Omega - \int_{\Gamma_D} \frac{d f(x)}{d x} n_x \, w \, d\Gamma - \int_{\Gamma_N} \frac{d f(x)}{d x} n_x \, w \, d\Gamma = 0
\]

In order to impose easily the dirichlet boundary conditions we'll use a $w$ such that $w=0$ on $\Gamma_D$. So the final weak form version is

\begin{equation}
\int_{\Omega} \frac{d f(x)}{d x} \frac{d w}{d x} \, d\Omega + \int_{\Omega} (6x-12) w \, d\Omega - \int_{\Gamma_N} \frac{d f(x)}{d x} n_x \, w \, d\Gamma = 0
\end{equation}

Then the problem using the weak form would be

\begin{equation}
\begin{array}{rl}
\displaystyle
\int_{\Omega} \frac{d f(x)}{d x} \frac{d w}{d x} \, d\Omega + \int_{\Omega} (6x-12) w \, d\Omega - \int_{\Gamma_N} \frac{d f(x)}{d x} n_x \, w \, d\Gamma = 0 & \textrm{ in } \mathbb{R} \\ \\
\displaystyle
f(x) = 0 & \textrm{ on } x = 1 \\ \\
\displaystyle
f'(x) n_x = \frac{d f(x)}{d x} n_x = 2 & \textrm{ on } x = 3 
\end{array}
\label{eq:Problem_WeakForm}
\end{equation}

\subsection{Discretization and the Galerkin method}

Now its time to do the discretization part and use the famous Galerkin method. First over a finite element method the solution will be interpolated like this

\begin{equation}
f(x) \approx \sum_{i=1}^n N_i(x_i) f(x_i)
\end{equation}

The $x_i$ are the location of the nodes that define the element and $N_i(x_i)$ are the shape functions (known as \textbf{trial functions}). Because we are in 1D our elements are bars and at least we need two nodes to define it.

MISSING FIGURE

So our finite element approximation will be, if we use two nodes

\[
f(x) \approx \sum_{i=1}^2 N_i(x_i) f(x_i) = N_1(x_1) f(x_1) + N_2(x_2) f(x_2)
\]

The Galerkin method implies to use $w=N_i(x)$. Considering this and the finite element approximation over an element we get

\begin{small}
\[
\int_{\Omega^e} \frac{d f(x)}{d x} \frac{d w}{d x} \, d\Omega + \int_{\Omega^e} (6x-12) w \, d\Omega - \int_{\Gamma_N^e} \frac{d f(x)}{d x} n_x \, w \, d\Gamma = 0
\]
\begin{multline}
\int_{\Omega^e} \frac{d \left( \sum_{i=1}^2 N_i(x_i)  f(x_i) \right) }{d x} \frac{d N_i(x)}{d x} \, d\Omega + \int_{\Omega^e} (6x-12) N_i(x)\, d\Omega \\
- \int_{\Gamma_N^e} \frac{d f(x)}{d x} n_x \, N_i(x) \, d\Gamma = 0
\end{multline}
\end{small}

You may be wondering why we ketp the $f(x)$ in the last term on the left side of the equation. Well this is because this integral is for the Neumann boundary conditions as you can see in equation \ref{eq:Problem_WeakForm}. This term will be not zero only when a Neumann condition is applied on the edges of a element. From the last formulae we get a system of equations, because $i=1,2$.

\begin{multline}
\begin{bmatrix}
\displaystyle
\int_{\Omega^e} \frac{d N_1(x)}{d x} \frac{d N_1(x)}{d x} \, d\Omega &
\displaystyle
\int_{\Omega^e} \frac{d N_1(x)}{d x} \frac{d N_2(x)}{d x} \, d\Omega \vspace{2mm} \\
\displaystyle
\int_{\Omega^e} \frac{d N_2(x)}{d x} \frac{d N_1(x)}{d x} \, d\Omega &
\displaystyle
\int_{\Omega^e} \frac{d N_2(x)}{d x} \frac{d N_2(x)}{d x} \, d\Omega
\end{bmatrix}
\begin{bmatrix}
y_1 \vspace{2mm}\\
y_2
\end{bmatrix}
= \\
\begin{bmatrix}
\displaystyle
-\bar{q}_1 \int_{\Gamma_N^e} N_1(x) \, d\Gamma \vspace{2mm}\\
\displaystyle
\bar{q}_2 \int_{\Gamma_N^e} N_2(x) \, d\Gamma
\end{bmatrix}
-
\begin{bmatrix}
\displaystyle
\int_{\Gamma_N^e} (6x-12) N_1(x) \, d\Gamma \vspace{2mm}\\
\displaystyle
\int_{\Gamma_N^e} (6x-12) N_2(x) \, d\Gamma
\end{bmatrix}
\label{eq:SystemEq_Element}
\end{multline}

This is the system of equations for each element. Note we substitute $f(x_i) = y_i$ and $\frac{d f(x_i)}{d x} = \bar{q}_i$. We can take out from the integral the $f(x_i)$ because his value does not change.

Another thing is $n_x$. In this case $n_x(x_1) = -1$ and $n_x(x_2)=1$, considering $x_1 \leq x_2$. The normal's direction is taken always pointing outwards.

\subsection{The shape functions}

If you can remember we set $w=N_i(x)$ but one condition that $w$ has to respect is $w=0$ on $\Gamma_D$. Also remember the finite element approximation $f(x) \approx \sum_{i=1}^2 N_i(x_i) f(x_i)$. One way to follow this rule is make each $N_i$ take value 1 in his node and 0 on the other ones.

In this case is easy to ``guess'' the expressions for $N_1$ and $N_2$ taking in account the rule

\begin{equation}
\begin{array}{cc}
\displaystyle
N_1(x) = \frac{x_2 - x}{x_2 - x_1} & 
\displaystyle
N_2(x) = \frac{x - x_1}{x_2 - x_1}
\end{array}
\end{equation}

The next step is substitute the expressions for the shape functions in eq. \ref{eq:SystemEq_Element}

\begin{multline*}
\begin{bmatrix}
\displaystyle
\int_{x_1}^{x_2} \frac{1}{(x_2-x_1)^2} \, dx &
\displaystyle
-\int_{x_1}^{x_2} \frac{1}{(x_2-x_1)^2} \, dx \vspace{2mm} \\
\displaystyle
-\int_{x_1}^{x_2} \frac{1}{(x_2-x_1)^2} \, dx &
\displaystyle
\int_{x_1}^{x_2} \frac{1}{(x_2-x_1)^2} \, dx
\end{bmatrix}
\begin{bmatrix}
y_1 \vspace{2mm}\\
y_2
\end{bmatrix}
= \\
\begin{bmatrix}
\displaystyle
\left. -\bar{q}_1 \frac{x_2 - x}{x_2 - x_1} \right]_{x_1}^{x_2} \vspace{2mm}\\
\displaystyle
\left. \bar{q}_2 \frac{x - x_1}{x_2 - x_1} \right]_{x_1}^{x_2}
\end{bmatrix}
-
\begin{bmatrix}
\displaystyle
\left. (6x-12) \frac{x_2 - x}{x_2 - x_1} \right]_{x_1}^{x_2} \vspace{2mm}\\
\displaystyle
\left. (6x-12) \frac{x - x_1}{x_2 - x_1} \right]_{x_1}^{x_2}
\end{bmatrix}
\end{multline*}

Doing the integrals we get

\begin{equation}
\begin{bmatrix}
\displaystyle
\frac{1}{x_2-x_1} &
\displaystyle
-\frac{1}{x_2-x_1}\vspace{2mm} \\
\displaystyle
-\frac{1}{x_2-x_1} &
\displaystyle
\frac{1}{x_2-x_1}
\end{bmatrix}
\begin{bmatrix}
y_1 \vspace{2mm}\\
y_2
\end{bmatrix}
=
\begin{bmatrix}
\displaystyle
\bar{q}_1 \vspace{2mm}\\
\displaystyle
\bar{q}_2
\end{bmatrix}
-
\begin{bmatrix}
\displaystyle
-(6x_1-12) \vspace{2mm}\\
\displaystyle
6(x_2 - 12)
\end{bmatrix}
\end{equation}

\section{Bar with axial force}


\end{document}

